\chapter{Crop image following the circle frame}

\section{Idea}
\begin{itemize}
    \item Determine the center point of the image. Choose the appropriate radius of the circle following the formula: ($min(width, height)$)
    \item The cropped image is obtained by extracting all pixels within the resulting region, bounded by the area where the Euclidean's distance \cite{euclide_distance} from center point to that pixel is less than or equal to the circle's radius.
\end{itemize}

\section{Processing}
\begin{enumerate}
    \item Convert image to \texttt{float32} to avoid overflow during the process.
    \item Determine the coordinate of center point and the radius following the above formula.
    \item Create image coordinate index grid for calculating Euclidean's distance.
    \item Create a Boolean mask matrix whose value is \texttt{true} when the Euclidean's distance is less than or equal to the radius; otherwise, it is \texttt{false}.
    \item A circle frame cropped image is obtained when collecting all the pixel which is valid using above mask matrix.
    \item Convert to \texttt{uint8} for correct display.
\end{enumerate}

\newpage

\section{Result}
\begin{itemize}
    \item Original image/Cropped image:
    \begin{center}
        \includegraphics[width=0.3\textwidth]{images/img.png}
        \includegraphics[width=0.3\textwidth]{images/img_circle_crop.png}
    \end{center} 
    \begin{center}
        \includegraphics[width=0.3\textwidth]{images/img2.jpg}
        \includegraphics[width=0.3\textwidth]{images/img2_circle_crop.jpg}
    \end{center} 
    \begin{center}
        \includegraphics[width=0.3\textwidth]{images/img3.jpg}
        \includegraphics[width=0.3\textwidth]{images/img3_circle_crop.jpg}
    \end{center} 
\end{itemize}